
Nesse capítulo, será apresentada a construção do operador grosso Multiescala. A construção desse tipo de operador tem diversas aplicações para a solução dos sistemas lineares apresentados no Capítulo \ref{ch:discretizacao}. Ele consiste basicamente em construir um operador grosso através do cálculo de funções de base em um grid gerado pelo acoplamento de elementos do grid fino. Pode ser utilizado como pré-condicionador \cite{casteletto}, como solver multinível semelhante aos solver multigrid ou ainda como aproximação para a solução original do problema. Os métodos multiescala tem sido aplicados com sucesso para problemas elípticos que é o caso do problema da elasticidade linear apresentado aqui. As vantagens do método discutido em \cite{thomashou} são que as funções de base multiescala tentam se adaptar às propriedades locais do operador, de forma que o operador grosso as conserve. As funções de base podem ser construídas através da solução de problemas independentes e, portanto, em paralelo.

O primeiro passo para a construção do operador multiescala é gerar um novo grid com menos elementos que o grid original do problema (grid fino), mas que ainda represente o mesmo domínio $\Omega$. Esse novo grid será chamado de grid grosso e as variáveis relacionadas com ele serão assinadas com o sobrescrito $H$. 
Assim grid grosso possui um conjunto $\tau_H$ de elementos onde cada elemento será uma aglomeração de elementos do grid fino. Por exemplo, a Figura \ref{fig:gridgrosso} apresenta um grid grosso $3\times 3$ construído a partir de um grid fino $7\times 7$. É importante perceber que não existe a necessidade de aglomerar a mesma quantidade de elementos do grid fino para se gerar um elemento do grid grosso, já que existem elementos formados por 9, 6 ou 4 elementos do grid fino. OS quadrados azuis destacam os nós que pertencem ao grid grosso e ao grid fino simultaneamente.

\begin{figure}[!htbp]
\centering
\includegraphics[width=6cm]{chap06/figs/grosso.png}
\caption{Exemplo de grid fino $7\time 7$ e um grid grosso $3\times 3$. O elemento inferior esquerdo é composto por 9 elementos do grid fino enquanto o elemento superior direito é composto com 4 elementos do grid fino.}
\label{fig:gridgrosso}

\end{figure}


No Capítulo \ref{ch:discretizacao} foi apresentada a discretização  para o problema de elasticidade linear através do método dos elementos finitos, com notação baseada em \cite{mbuck}. Naquele capítulo, a função desejada foi aproximada em um espaço $V^h$ formado por funções de base chapéu, no domínio $\Omega$ do problema. Agora vamos encontrar a solução do problema em um espaço grosso $V^{MS}$, tal que $V^{MS} \subset V^h$. 



Para estabelecer o espaço $V^{MS}$, precisamos definir as suas funções de base geradoras. 
Novamente, para distinguir o nó pertencente ao grid grosso dos seus graus de liberdade, o seguinte conjunto com graus de liberdade é criado:


\begin{equation}\label{eq:dheq}
    D^H = \{ p^{(m)} \in D^h : x^p \in \Sigma_H, m=1,2\}.
\end{equation}

\suge{acho que vc deveria esclarecer aqui cada um dos elementos da equação \eqref{eq:dheq}, mesmo que já tenha feito isso em algum capítulo anterior}

\section{Cálculo do NNZ do operador de prolongamento}\label{sec:complexProlong}

Em um método multiescala, além da construção de um operador grosso,  precisa-se mover os vetores entre os espaços fino e grosso  Essas movimentações são realizadas através de multiplicação por operadores especiais, $P$ e $P^T$. Essa seção descreve a complexidade destas operações.

As dimensões do operador de prolongamento dependem dos graus de liberdade do operador fino e do  grosso, valendo $\freedomfine \times \freedomcoarse$. Conforme visto no Capítulo \ref{ch:sistemas}, a multiplicação de uma matriz esparsa por um vetor é da ordem do número de elementos  não nulos da matriz, assim, apesar de um maior engrossamento do grid reduzir o valor de $n_u^H$,  não necessariamente ocorre  uma redução de elementos não nulos de $P$. \suge{Talvez uma forma simples de motivar esta ideia, seja dizer que todos os elementos a matriz $A$ tem que ser multiplicados por algum elemento da matriz de restrição, não podendo ficar nenhum de fora, sendo assim, mesmo que ser reduza o número de linhas do operador de restrição, a quantidade de elementos não nulos não pode ser alterada. O que ocorre é que cada linha do operador de restrição, ou coluna do de extensão, fica mais densa quando se diminui a dimensão do espaço grosso}

Considerando um grid fino com $\numelementsxfine \times \numelementsyfine$, um grid grosso onde cada elemento tem dimensões $q_x \times q_y$ \suge{neste caso você está considerando que os elementos grossos são formados sempre pelo mesmo número de elementos, talvez seja o caso de dizer que você vai fazer isso mais lá em cima, apesar de isso não ser necessário} e, ainda, que mesmo os nós de fronteira com condição de contorno de Dirichlet não são removidos da matriz de rigidez, então a quantidade de não zeros do operador de prolongamento ($nnz_p$) pode ser calculada contando os nós de três conjuntos:  nós interiores, nós no vértice e os nós na aresta.  A Figura \ref{fig:gridCompleto} mostra cada um desse tipo de nós (em vermelho nó no vértice, em azul nó na aresta e em preto nó interior).


\begin{figure}[h]
\center
\subfigure[  ]{\includegraphics[width=0.45\textwidth]{chap06/figs/gridCompleto.jpeg}\label{fig:gridCompleto}}
\qquad
\subfigure[ ]{\includegraphics[width=0.45\textwidth]{chap06/figs/noInterior.jpeg}\label{fig:noInterior}}
\subfigure[ ]{\includegraphics[width=0.45\textwidth]{chap06/figs/noVertice.jpeg}\label{fig:noVertice}}
\subfigure[ ]{\includegraphics[width=0.45\textwidth]{chap06/figs/noAresta.jpeg}\label{fig:noAresta}}

\caption{Comparação da solução do grid fino com a solução do grid grosso.  }
\label{fig:verticesTypes}
\end{figure}

A quantidade de cada um desses nós na malha grossa é apresentado abaixo.

\begin{itemize}
    \item Nós interiores: $(\frac{n_x}{q_x} - 1) (\frac{n_y}{q_y} - 1)$
    \item Nós de aresta em $\Gamma_l$ e $\Gamma_r$: $2 ( \frac{n_y}{q_y} - 1)$
    \item Nós de aresta em $\Gamma_t$ e $\Gamma_b$: $2 ( \frac{n_x}{q_x} - 1)$
    \item Nós vértices: 4 
\end{itemize}


Cada nó tem duas funções de base associadas uma para cada grau de liberdade (x e y). Uma função de base de um vértice interior tem suporte \suge{como vc está definindo suporte? Em matemática, são os pontos do domínio onde a função é diferente de zero e, me parece, que não é isso que vc está dizendo. Acho que vc está falando de que haverá esse número de funções de base que estarão relacionadas a um nó interior. Ou seja a função definida no próprio nó e as funções definidas nos nós vizinhos, interiores ou não. Essa relação dá o número de elementos não nulos da linha desse nó interior} em $(2q_x+1)(2q_y+1)$ outros nós, como mostra a \ref{fig:noInterior}. Além disso, ela afeta os dois graus de liberdade de cada um desses nós, então cada um deles contribui com $4(2q_x+1)(2q_y+1)$ \perg{por que 4} não zeros para o prolongamento (uma implementação mais cuidadosa do método pode economizar as bordas do suporte pois as funções se anulam). De maneira análoga, os nós vértices contribuem com $ 4 (q_x+1)(q_y+1)$, os nós aresta em $\Gamma_t$ e $\Gamma_b$ contribuem com $ 4 (2q_x+1)(q_y+1) $ e, finalmente, os nós arestas  $\Gamma_l$ e $\Gamma_r$ e contribuem com $ 4(q_x+1)(2q_y+1) $. Assim, para encontrar a quantidade total de não zeros do operador de prolongamento, basta multiplicar as quantidades de cada um dos nós pela duas contribuições conforme a   \eqref{eq:nnzRaw}. 


\begin{equation} \label{eq:nnzRaw}
\begin{aligned}
    nnz_P = & 4(2q_x+1)(2q_y+1)  (\frac{n_x}{q_x} - 1) (\frac{n_y}{q_y} - 1)   \\ 
            & + 4 (q_x+1)(2q_y+1)  2 ( \frac{n_y}{q_y} - 1) +  4 (2q_x+1)(q_y+1)  2 (\frac{n_x}{q_x} - 1) \\
            & +  4(q_x+1)(q_y+1) 4 
\end{aligned}
\end{equation}

Que pode ser modificada para a \eqref{eq:nnzRaw2},

\begin{equation} \label{eq:nnzRaw2}
\begin{aligned}
    nnz_P = &   4 (2+\frac{1}{q_x})(2 + \frac{1}{q_y})  (n_x - q_x) (n_y - q_y) \\ 
            & + 8 (q_x+1)(2 + \frac{1}{q_y})  (n_y - q_y) \\
            & + 8 (2+\frac{1}{q_x})(q_y+1) (n_x - q_x) \\
            & + 16 (q_x+1)(q_y+1)  
\end{aligned}
\end{equation}

Assim, se $q_x$ e $q_y$ são de ordem $O(1)$ o primeiro termo da soma é da ordem de $O(n_x \times n_y)$. Caso $q_x$ e $q_y$ sejam de ordem $O(n_x)$ e $O(n_y)$, então o último termo da soma é da ordem $O(n_x \times n_y )$. \suge{não vejo muito sentido em $q_x$ ser $O(1)$, pois $q_x$ e $n_x$ vão sempre ter uma relação direta e, neste caso, o conceito de ordem esconde a constante de proporcionalidade e não ajuda a definir bem a complexidade do problema. $q_x$ e $n_x$ não teriam relação caso um fosse fixado e outro alterado, que é a situação mostrada na Figura \ref{fig:nnzGrafico}}
Se $q_x$ é $O(n_x)$ e $q_y$ é $O(1)$ então o segundo termo é de ordem $O(n_x \times n_y)$, analogamente para o caso contrário.
De toda forma, a quantidade de não zeros do prolongamento é da ordem do tamanho do grid $n_x \times n_y$ que é um fato importante, pois a multiplicação pelo prolongamento faz parte do processo de utilização do pré-condicionador multiescala e esse preço sempre terá que ser pago.\suge{tudo bem isso, é verdade, pois o operador de prolongamento tem $n_x \times n_y$ linhas, mas esse fato deve ser ponderado pelo número de colunas do operador} O valor limite também dos não zeros é representado pelo último termo da soma $16(q_x+1)(q_y+1)$. \suge{acho que vale a pena provar essa afirmação e não deixá-la apenas visual}

Um gráfico representando    \eqref{eq:nnzRaw2} é mostrado na Figura \ref{fig:nnzGrafico}, no eixo y é apresentado $nnz_P$ enquanto no eixo x é apresentado o engrossamento da malha que tem a mesma proporção em x e y ($q_x  = q_y$) para um grid de $1024 \times 1024$. Pelo gráfico é possível ver que o $nnz_P$ tende já em um engrossamento $16 \times 16$.


\begin{figure}[!htbp]
\centering
\includegraphics[width=8cm]{chap06/figs/nnzProlongamento.png}
\caption{Quantidade de não zeros do prolongamento $nnz_P$  em função do engrossamento da malha. A linha pontilhada mostra o valor de $16(n_x+1)(n_y+1)$.}
\label{fig:nnzGrafico}
\end{figure}
