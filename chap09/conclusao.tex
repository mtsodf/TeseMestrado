
Essa dissertação apresentou a teoria necessária para a utilização do método multiescala como pré-condicionador para sistemas lineares advindos de simulações geomecânicas elásticas. Como resultados foram apresentadas comparações entre a abordagem de combinar pré-condicionadores de forma aditiva e multiplicativa que mostram que, para fatores de engrossamento grandes o suficientes, pode-se obter soluções mais rápidas em torno de 10\% utilizando o pré-condicionador aditivo. Além disso, foram apresentadas comparações utilizando o método do gradiente conjugado com pré-condicionador  multiescala, ILU e multigrid para modelos sintéticos e também para casos reais. A comparação entre o multiescala e o ILU, desconsiderando o tempo de montagem do operador, mostrou que o multiescala reduz o número de iterações e encontra soluções mais rapidamente que o ILU.  A comparação entre o multigrid e multiescala mostrou que em apenas um dos casos o multigrid desempenhou pior com relação ao número de iterações mas em todos os outros ele realizou uma quantidade próxima de iterações, porém, a configuração utilizada tem uma iteração mais pesada que o multiescala com relação a métrica do número de operações realizadas tornando o multiescala uma opção mais atrativa. É possível ver também que na simulações dos casos D e E o multigrid começa com uma inclinação elevada na redução do resíduo e a partir de certo ponto a inclinação muda para uma redução menor do resíduo, enquanto que a inclinação do método multiescala permanece mais constante. Uma possível tentativa para atingir reduções elevadas no resíduo nas primeiras iterações do método multiescala é a utilização do método multiescala multi-nível.

Visto o bom resultado do método para os sistemas lineares geomecânicos aqui apresentados, uma implementação paralela e algébrica tem grandes possibilidades de reduzir o tempo de simulação em modelos de simulação geomecânica em três dimensões. Além disso, como o operador multiescala também possui uma dimensão bem menor que a do operador do grid original, ele pode ser pensando com um pré-condicionador grosseiro geral de todo o modelo resolvendo dessa forma o problema do aumento do número de partições quanto mais o domínio é dividido. Dessa forma, essas ideias podem ajudar a melhorar ainda mais os resultados obtidos em \citet{geomecrio} tornando as execuções do simulador geomecânico da Petrobras ainda mais rápidas. Essas novas implementações tem diversos desafios em relação ao paralelismo como a coordenação da solução dos problemas locais em paralelo, implementação de produtos matriz-matriz esparsas para  cálculo do operador grosso, comunicação dos resultados de problemas locais nas fronteiras do processos, dentre outros.


