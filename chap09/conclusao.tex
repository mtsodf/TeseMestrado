
Essa dissertação apresentou a teoria necessária para a construção de um simulador de geomecânica elástico utilizando o método dos elementos finitos e método multiescala como pré-condicionador para sistemas lineares. Como resultados foram apresentadas comparações entre a abordagem de combinar pré-condicionadores de forma aditiva e multiplicativa que mostram que, para fatores de engrossamento grandes o suficientes, pode-se obter soluções em torno de 10\% a 15\%  mais rápidas utilizando o pré-condicionador aditivo. Além disso, foram apresentadas comparações utilizando o método do gradiente conjugado com pré-condicionador  multiescala, ILU e multigrid para modelos sintéticos e também para casos reais. A comparação entre o multiescala e o ILU mostrou que o primeiro reduz o número de iterações e encontra soluções mais rapidamente que o ILU mostrando speed-ups de até 3,8 vezes. A comparação entre o multigrid e multiescala mostrou que apenas no caso E o multigrid realizou mais iterações enquanto que nos outros a quantidade de iterações foi similar. Porém, com relação a métrica do número de operações realizadas as relaxações dos vários níveis do multigrid  tem um custo bem maior por iteração, tornando o multiescala uma opção mais atrativa.  Outro ponto importante é que os resultados mostraram também a eficácia do método para a simulação de modelos com dados reais de campos que possuem propriedades heterogêneas e grids mais complexos. 
%É possível ver também que na simulações dos casos D e E o multigrid começa com uma inclinação elevada na redução do resíduo e a partir de certo ponto a inclinação muda para uma redução menor do resíduo, enquanto que a inclinação do método multiescala permanece mais constante. Uma possível tentativa para atingir reduções elevadas no resíduo nas primeiras iterações do método multiescala é a utilização do método multiescala multi-nível.

Visto o bom resultado do método para os sistemas lineares geomecânicos, uma implementação paralela e algébrica tem grandes possibilidades de reduzir o tempo de simulação em modelos de simulação geomecânica em três dimensões. Além disso, como o operador multiescala também possui uma dimensão bem menor que a do operador do grid original, ele pode ser pensando com um pré-condicionador grosseiro geral de todo o modelo resolvendo dessa forma o problema do aumento do número de partições quanto mais o domínio é dividido. Com essas ideias pode-se melhorar ainda mais os resultados obtidos em \citet{geomecrio} tornando as execuções do simulador geomecânico da Petrobras ainda mais rápidas. Essas novas implementações tem diversos desafios em relação ao paralelismo como a coordenação da solução dos problemas locais em paralelo, implementação de produtos matriz-matriz esparsas para  cálculo do operador grosso, comunicação dos resultados de problemas locais nas fronteiras do processos, dentre outros.


