\newcommand{\wbalTwo}[2][Wikimedia]{
  This is the Wikibook about LaTeX
  supported by {#1} and {#2}!}


\newcommand{\dx}[1][ ]{\frac{\partial {#1}}{\partial x}}
\newcommand{\dy}[1][ ]{\frac{\partial {#1}}{\partial y}}
\newcommand{\dz}[1][ ]{\frac{\partial {#1}}{\partial z}}
\newcommand{\dxi}[1][ ]{\frac{\partial {#1}}{\partial \xi}}
\newcommand{\deta}[1][ ]{\frac{\partial {#1}}{\partial \eta}}
\newcommand{\dzeta}[1][ ]{\frac{\partial {#1}}{\partial \zeta}}

\newcommand{\dxk}[1][ ]{\frac{\partial {#1}}{\partial x_k}}

\newcommand{\Tau}{\mathrm{T}}

\newcommand{\poisson}{\upsilon}

%Deslocamentos

\newcommand{\uvetor}{
\begin{bmatrix}
u_x \\ u_y \\ u_z
\end{bmatrix}
}

%Deformacoes
\newcommand{\exx}{\epsilon_{xx}}
\newcommand{\eyy}{\epsilon_{yy}}
\newcommand{\ezz}{\epsilon_{zz}}
\newcommand{\exy}{\gamma_{xy}}
\newcommand{\exz}{\gamma_{xz}}
\newcommand{\eyx}{\gamma_{yx}}
\newcommand{\eyz}{\gamma_{yz}}
\newcommand{\ezx}{\gamma_{zx}}
\newcommand{\ezy}{\gamma_{zy}}


%Definicao das tensoes efetivas
\newcommand{\sxx}{\sigma^\prime_{xx}}
\newcommand{\syy}{\sigma^\prime_{yy}}
\newcommand{\szz}{\sigma^\prime_{zz}}
\newcommand{\sxy}{\sigma^\prime_{xy}}
\newcommand{\sxz}{\sigma^\prime_{xz}}
\newcommand{\syx}{\sigma^\prime_{yx}}
\newcommand{\syz}{\sigma^\prime_{yz}}
\newcommand{\szx}{\sigma^\prime_{zx}}
\newcommand{\szy}{\sigma^\prime_{zy}}

\newcommand{\evetor}{
\begin{bmatrix}
\exx
\\
\eyy
\\
\ezz
\\
\exy
\\
\exz
\\
\eyz
\end{bmatrix}
}


% Definicao das tensoes totais
\newcommand{\stxx}{\sigma_{xx}}
\newcommand{\styy}{\sigma_{yy}}
\newcommand{\stzz}{\sigma_{zz}}
\newcommand{\stxy}{\sigma_{xy}}
\newcommand{\stxz}{\sigma_{xz}}
\newcommand{\styx}{\sigma_{yx}}
\newcommand{\styz}{\sigma_{yz}}
\newcommand{\stzx}{\sigma_{zx}}
\newcommand{\stzy}{\sigma_{zy}}

\newcommand{\stvetor}{
\begin{bmatrix}
\stxx
\\
\styy
\\
\stzz
\\
\stxy
\\
\stxz
\\
\styz
\end{bmatrix}
}

\newcommand{\stensor}{
\begin{bmatrix}
\stxx & \stxy & \stxz \\
\styx & \styy & \styz \\
\stzx & \stzy & \stzz
\end{bmatrix}
}

\newcommand{\sstensor}{
\begin{bmatrix}
\dx[\stxx] + \dy[\stxy] + \dz[\stxz] \\
\dx[\styx] + \dy[\styy] + \dz[\styz] \\
\dx[\stzx] + \dy[\stzy] + \dz[\stzz]
\end{bmatrix}
}

\newcommand{\Dx}{
\Delta x
}
\newcommand{\Dy}{
\Delta y
}

\newcommand{\Dz}{
\Delta z
}


%Matriz de Elasticidade
\newcommand{\elasticinv}{
\frac{1}{E}
\begin{bmatrix}
        1 & -\upsilon & -\upsilon &             0 &             0 &             0 \\
-\upsilon &         1 & -\upsilon &             0 &             0 &             0 \\
-\upsilon & -\upsilon &         1 &             0 &             0 &             0 \\
        0 &         0 &         0 & 2(1+\upsilon) &             0 &             0 \\
        0 &         0 &         0 &             0 & 2(1+\upsilon) &             0 \\
        0 &         0 &         0 &             0 &             0 & 2(1+\upsilon)
\end{bmatrix}
}


\newcommand{\elastic}{
\frac{E}{(1+\upsilon)(1-2\upsilon)}
\begin{bmatrix}
1 - \upsilon &     \upsilon &     \upsilon &                     0 &                     0 &                     0 \\
    \upsilon & 1 - \upsilon &     \upsilon &                     0 &                     0 &                     0 \\
    \upsilon &     \upsilon & 1 - \upsilon &                     0 &                     0 &                     0 \\
           0 &            0 &            0 & \frac{1-2\upsilon}{2} &                     0 &                     0 \\
           0 &            0 &            0 &                     0 & \frac{1-2\upsilon}{2} &                     0 \\
           0 &            0 &            0 &                     0 &                     0 & \frac{1-2\upsilon}{2}
\end{bmatrix}
}


% Operador S

\newcommand{\sopwithoutb}[1][ ]{
\dx[#1] &        0 &   0      \\
      0 & \dy[#1]  &   0      \\
      0 &        0 & \dz[#1]  \\
\dy[#1] & \dx[#1]  &   0      \\
\dz[#1] &        0 & \dx[#1]  \\
      0 & \dz[#1]  & \dy[#1]
}

\newcommand{\sopnabla}{\nabla_S}

\newcommand{\sop}[1][ ]{
\begin{bmatrix}
\dx[#1] &        0 &   0      \\
      0 & \dy[#1]  &   0      \\
      0 &        0 & \dz[#1]  \\
\dy[#1] & \dx[#1]  &   0      \\
\dz[#1] &        0 & \dx[#1]  \\
      0 & \dz[#1]  & \dy[#1]
\end{bmatrix}
}

\newcommand{\soptwod}[1][ ]{
\begin{bmatrix}
\dx[#1] &        0   \\
      0 & \dy[#1]    \\
\dy[#1] & \dx[#1]    \\
\end{bmatrix}
}

\newcommand{\sopt}[1][ ]{
\begin{bmatrix}
\dx[#1] &       0 &   0     & \dy[#1] & \dz[#1] &   0 \\
      0 & \dy[#1] &   0     & \dx[#1] &       0 & \dz[#1] \\
      0 &       0 & \dz[#1] &       0 & \dx[#1] & \dy[#1]
\end{bmatrix}
}

% Integral em omega

\newcommand{\omeint}[2][\Omega]{\int\limits_{#1} #2 \, d#1}
\newcommand{\omeeint}[1]{\omeint[\Omega^e]{#1}}

\newcommand{\phiix}{\varphi_i(\mathbf{x})}
\newcommand{\phijx}{\varphi_j(\mathbf{x})}

\newcommand{\aepp}[2]{a_e(\varphi_{#1}, \varphi_{#2})}

\newcommand{\aepplocal}[2]{a_e(\varphi^e_{#1}, \varphi^e_{#2})}

\newcommand{\aelinha}[1][i]{\aepp{#1}{1} & \aepp{#1}{2} & \cdots & \aepp{#1}{nn}}

\newcommand{\aelocallinha}[1][i]{\aepplocal{#1}{1} & \aepplocal{#1}{2} & \cdots & \aepplocal{#1}{8}}

\newcommand{\aemat}{
\begin{bmatrix}
\aelinha[1] \\
\aelinha[2] \\
\vdots & \vdots & \ddots & \vdots \\
\aelinha[nn]
\end{bmatrix}}

\newcommand{\aematelem}{
\begin{bmatrix}
\aelocallinha[1] \\
\aelocallinha[2] \\
\vdots & \vdots & \ddots & \vdots \\
\aelocallinha[8] \\
\end{bmatrix}
}

\newcommand{\libvetor}{
 \begin{bmatrix}
  u^{1}_x \\
  u^{1}_y \\
  u^{1}_z \\

  \vdots\\
  u^{nn}_x \\
  u^{nn}_y \\
  u^{nn}_z
 \end{bmatrix}
}



\newcommand{\ldint}[1][i]{(\mathbf{F}, \varphi_{#1})}


%Matrizes Griffiths
\newcommand{\der}{\begin{bmatrix}
\dxi[\phi^1]   & \dxi[\phi^2] & \dxi[\phi^3] & \dxi[\phi^4]      \\
\deta[\phi^1]  & \deta[\phi^2] & \deta[\phi^3] & \deta[\phi^4]    \\
\end{bmatrix}
}

\newcommand{\deriv}{\begin{bmatrix}
\dx[\phi^1]   & \dx[\phi^2] & \dx[\phi^3] & \dx[\phi^4]    \\
\dy[\phi^1]   & \dy[\phi^2] & \dy[\phi^3] & \dy[\phi^4]
\end{bmatrix}
}

\newcommand{\entradaae}[2]{
 (S^T N^{#1})_{x=p_k} D (S N^{#2})_{x=p_k}
}


\newcommand{\preconadd}{
    M^{-1}=M^{-1}_{h} + M^{-1}_{ms}
}

\newcommand{\preconmult}{
    M^{-1}=M^{-1}_{h} + M^{-1}_{ms}(I-K_hM^{-1}_{h})
}

\newcommand{\qtdfreedomfine}{n_u^h}
\newcommand{\essentialfine}{n_{\bar{u}}^h}

\newcommand{\qtdfreedomcoarse}{n_u^H}
\newcommand{\essentialcoarse}{n_{\bar{u}}^H}

\newcommand{\numelementsxfine}{n_x}
\newcommand{\numelementsyfine}{n_y}
\newcommand{\numelementsxcoarse}{n^H_x}
\newcommand{\numelementsycoarse}{n^H_y}


\newcommand{\basefunctionfine}[1][i]{ {\mathbf{N}_{#1}^h} }
\newcommand{\basefunctionelem}[1][i]{ {\mathbf{N}_{#1}^e} }
\newcommand{\basefunctionfineessential}{\mathbf{N}_{i+\qtdfreedomfine}^h}

\newcommand{\basefunctioncoarse}[1][i]{ {\mathbf{N}_{#1}^H} }




\newcommand{\sobolev}{H^1(\Omega)}

\newcommand{\trial}{\mathcal{V}(\Omega)}

\newcommand{\trialaprox}{\mathcal{V}^h}

\newcommand{\trialdef}{\{  \mathbf{w} | \mathbf{w} \in \sobolev ^2 , \mathbf{w} = 0 \text{ em } \Gamma_u \}}

\newcommand{\test}{\mathcal{S}(\Omega)}

\newcommand{\testaprox}{\mathcal{S}^h}

\newcommand{\testdef}{\{  \mathbf{u} | \mathbf{u} \in \sobolev ^2 , \mathbf{u} = \bar{u} \text{ em } \Gamma_u \}}

\newcommand{\qtdnos}{n_{\text{nós}}}

\newcommand{\kij}[2]{(\sopnabla \basefunctionfine[#1])^T D \sopnabla \basefunctionfine[#2]}

\newcommand{\kije}[2]{(\sopnabla \basefunctionelem[#1])^T D \sopnabla \basefunctionelem[#2]}


\newcommand{\intbase}[1]{\int_{-1}^{1}\int_{-1}^{1} #1 d\xi d\eta }