
Nesse capítulo, serão apresentados as equações que regem os fenômenos geomecânicos. O primeiro trabalho famoso na modelagem de tensões no solo é Terzaghi que desenvolveu a teoria para uma dimensão. Mais tarde, Biot generalizou essa teoria para três dimensões e essas são as equações que serão apresentadas. Detalhes das duas teorias podem ser encontradas em \citet{CompGeomec}. Como o intuito do trabalho é avaliar o método multiescala aplicado para matrizes dessa natureza, o caso aqui considerado será o de elasticidade linear onde a rocha não é tensionada até que apresente falhas.



\section{Tensor de Tensões}

Para representar a tensão total em um ponto da rocha, é utilizado um tensor de tensões de segunda ordem como mostra \eqref{eq:tensordetensoes}.

\begin{equation} \label{eq:tensordetensoes}
\mathbf{\st} =
    \begin{bmatrix}
    \stxx & \stxy & \stxz \\
    \styx & \styy & \styz \\
    \stzx & \stzy & \stzz
    \end{bmatrix}
\end{equation}

O primeiro subscrito do tensor representa a face que a tensão está sendo aplicada, enquanto o segundo representa a direção da tensão. A Figura \ref{fig:tensoesx} mostra as componentes com o com primeiro subscrito $x$.


\begin{figure}[!htbp]
\centering
\includegraphics[width=5cm]{chap01/figs/tensor.png}
\caption{Tensões $\sigma_{x.}$ representadas graficamente.}
\label{fig:tensoesx}
\end{figure}

Ao aplicar a condição de equilíbrio do momento, chega-se a conclusão que $\stxy=\styx$, $\stxz=\stzx$ e $\styz=\stzy$. Dessa maneira, para a representação desse tensor, são necessários guardar apenas seis valores. Pode-se considerar então a tensão como o vetor apresentado \eqref{eq:tensor6} essa notação é chamada de notação de Voigt e é a bastante utilizada nas implementações de elementos finitos, por exemplo, nas formulações apresentadas por \citet{hughes} e \citet{jacob}.


\begin{equation}
\st^T = \begin{bmatrix}
\stxx & \styy & \stzz & \stxy & \stxz & \styz
\end{bmatrix}
\label{eq:tensor6}
\end{equation}


\section{Teoria da Consolidação}

Para um certo elemento de volume $\Delta x\Delta y \Delta z$ pode-se escrever as equações de equilíbrio para cada uma dessas direções. A Figura \ref{fig:equilibrio} apresenta todas as tensões relativas à direção x.

\begin{figure}[!htbp]
\centering
\includegraphics[width=8cm]{chap01/figs/equilibrio.png}
\caption{Tensões na direção x ($\sigma_{.x}$) representadas graficamente.  Adaptado de \cite{CompGeomec}.}
\label{fig:equilibrio}
\end{figure}


As forças atuantes no cubo são calculadas a partir da tensão multiplicada pela área de atuação. Sobre o valor $f_x$, ele, juntamente com os termos $f_y$ e $f_z$  são termos gravitacionais e representam a força peso do volume $\Dx\Dy\Dz$. Como apresentado em \citet{xiuligai}, $\mathbf{f} = [\rho_s(1-\phi) + \phi(\rho_o S_o + \rho_w S_w + \rho_g S_g )] \mathbf{g}$ onde $\phi$ é a porosidade, $S_o$, $S_w$ e $S_g$ as saturações de óleo, água e gás e $\rho_o$, $\rho_w$, $\rho_g$, $\rho_s$  as densidades do óleo, água, gás e rocha e $\mathbf{g}$ o vetor gravidade. O termo gravitacional será considerado constante ao longo da simulação, fazendo com que as modificações no equilíbrio sejam apenas relativas à mudança de pressão. 

O equilíbrio de forças na direção x pode ser escrito como apresentado em \eqref{eq:equilibrio0}. 


\begin{multline} \label{eq:equilibrio0}
   (\stxx - \stxx - \Delta \stxx) \Dy\Dz + (\styx - \styx - \Delta \styx)\Dx\Dz  +\\
   + (\stzx - \stzx - \Delta\stzx) \Dx\Dy + f_x\Dx\Dy\Dz = 0
\end{multline}


\begin{equation}
 \Delta \stxx \Dy\Dz + \Delta \styx\Dx\Dz + \Delta\stzx \Dx\Dy - f_x\Dx\Dy\Dz = 0
\end{equation}

Dividindo todos os termos por $\Dx\Dy\Dz$.


\begin{equation}
\frac{ \Delta \stxx }{ \Dx } + \frac{\Delta \styx}{ \Dy } + \frac{\Delta\stzx}{ \Dz } - f_x = 0
\end{equation}

Fazendo $\Dx \rightarrow 0$, $\Dy \rightarrow 0 $ e $\Dz \rightarrow 0$

\begin{equation}
\dx[\stxx] + \dy[\styx] + \dz[\stzx] - f_x = 0
\end{equation}

Analogamente, para as direções y e z pode-se encontrar a equação de equilíbrio também, montando o sistema \eqref{eq:equilibrio1}.

\begin{equation}
\label{eq:equilibrio1}
\left\{\begin{matrix}
 \dx[\stxx] + \dy[\styx] + \dz[\stzx] - f_x & = & 0\\
 \dx[\stxy] + \dy[\styy] + \dz[\stzy] - f_y & = & 0\\
 \dx[\stxz] + \dy[\styz] + \dz[\stzz] - f_z & = & 0
\end{matrix}\right.
\end{equation}


As tensões apresentadas em \eqref{eq:equilibrio1} são as tensões totais que atuam no cubo infinitesimal. Acontece que, como mostra a Figura \ref{fig:rochaComFluido}, os reservatórios de petróleo possuem fluído no volume poroso da rocha (óleo, água e gás) e, portanto, parte da tensão total será suportada pelo fluído e parte será suportada pelos grãos da rocha. Como o fluído não oferece resistência ao cisalhamento, ele suporta apenas a parte das tensões $\stxx$, $\styy$ e $\stzz$.

\begin{figure}[!htbp]
\centering
\includegraphics[width=7cm]{chap01/figs/fluido_rocha_tensoes.png}
\caption{Representação de meio poroso. Grãos amarelos representando a matriz da rocha e fluído representado pela cor azul.}
\label{fig:rochaComFluido}
\end{figure}

Conforme mostrado pela teoria da consolidação de poroelasticidade, desenvolvida por Biot para três dimensões, a tensão efetiva na rocha ($\stef$) e a tensão total são relacionadas por \eqref{eq:tensaoefetiva}, conforme mostrado em \citet{ResGeomec}.


\begin{equation}
\label{eq:tensaoefetiva}
\left\{\begin{matrix}
 \sxx = \stxx - \alpha P_p \\
 \syy = \styy - \alpha P_p \\
 \szz = \stzz - \alpha P_p
\end{matrix}\right.
\end{equation}


Onde $\sxx$, $\syy$  e $\szz$ são as tensões efetivas, $\alpha$ é o coeficiente de Biot e $P_p$ a pressão de poros. O coeficiente de Biot representa o quanto a pressão de poros do fluído suporta a tensão total na rocha e está no intervalo $\alpha \in [0,1]$.

As equações \eqref{eq:equilibrio1} podem ser reescritas como \eqref{eq:equilibrio} substituindo as tensões totais ($\st$) pela tensões efetivas na rocha ($\stef$).

\begin{equation}
\label{eq:equilibrio}
\left\{\begin{matrix}
\dx[\sxx]  + \dy[\syx] + \dz[\szx] - \dx[\alpha P_p] - f_x   = 0
\\
\dx[\sxy]  + \dy[\syy] + \dz[\szy] - \dy[\alpha P_p] - f_y   = 0
\\
\dx[\sxz]  + \dy[\syz] + \dz[\szz] - \dz[\alpha P_p] - f_z   = 0
\end{matrix}\right.
\end{equation}

Ou ainda, escrevendo \eqref{eq:equilibrio} utilizando os operadores divergente, gradiente e $\mathbf{f}^T=\begin{bmatrix}f_x & f_y & f_z\end{bmatrix}$ pode-se chegar em
\eqref{eq:equilibrio_matriz}.

\begin{equation}
\label{eq:equilibrio_matriz}
\nabla \cdot \stef - \nabla \cdot \alpha \mathbf{I} P_p - \mathbf{f} = 0
\end{equation}

Por motivos de implementação mais eficiente, menor uso de memória e utilização de operações mais simples (multiplicação de matrizes por vetores), a Equação \eqref{eq:equilibrio_matriz} pode ser escrita na notação de Voigt considerando as definições conforme \eqref{eq:definicoesvoigt}.

\begin{equation}
\label{eq:equilibrio_final}
\sopnabla^T \stef - \sopnabla^T\alpha \mathbf{m}  P_p - \mathbf{f} = 0
\end{equation}

Onde,
\begin{equation}
\begin{matrix}\label{eq:definicoesvoigt}
\stef = \begin{bmatrix}
\sxx
\\
\syy
\\
\szz
\\
\sxy
\\
\sxz
\\
\syz
\end{bmatrix}
&

;

&

\mathbf{f} = \begin{bmatrix}
f_{x}
\\
f_{y}
\\
f_{z}
\end{bmatrix}
&
;
&

\mathbf{m} = \begin{bmatrix} 1 \\ 1 \\ 1 \\ 0 \\ 0 \\ 0\end{bmatrix}

&
;

&
\sopnabla = \sop
\end{matrix}
\end{equation}



\subsection{Relações Constitutivas}

A Equação \eqref{eq:equilibrio_final} apresenta o equilíbrio em função das tensões. Porém, pelo grande número de variáveis dessa equação, é necessário colocar em função das variáveis de deslocamento para que seja possível resolve-la. O campo de deslocamentos será representado pela variável $\deslocamento = [u_x \quad u_y \quad u_z]^T$. A deformação ($\deformacao$) se relaciona com o deslocamento de acordo com \eqref{eq:defor_desloc}. 

\begin{equation}
\label{eq:defor_desloc}
\deformacao = \sopnabla \deslocamento
\end{equation}


Já a relação entre a deformação e as tensões é definida como relação constitutiva de acordo com \citet{ResGeomec}. Diferentes tipos de relações constitutivas podem ser utilizadas para representar essa relação entre tensão e deformação. A Figura \ref{fig:stress_strain} mostra dados de um teste típico de tensão-deformação em uma rocha bem cimentada. Nesse caso, é importante notar que existe um comportamento linear dominante na parte central do gráfico antes da falha. Essa região onde as deformações e tensões se relacionam linearmente é chamada de elástica, e será a região de estudo desse trabalho. 


\begin{figure}[!htbp]
\centering
\includegraphics[width=8cm]{chap01/figs/stress_strain.png}
\caption{Teste de laboratório tensão-deformação para uma rocha bem cimentada. Adaptada de \citet{ResGeomec}.}
\label{fig:stress_strain}
\end{figure}


Para o caso de elasticidade linear, a relação entre tensões e deformação é dada por uma forma simples  que é a Lei de Hooke Generalizada apresentada em \eqref{eq:hooke}. Essa equação é a mesma que apresentada a para estudos de mecânica dos sólidos clássicos.

\begin{equation}{
\label{eq:hooke}
\stef = \mathbf{D} \deformacao
}
\end{equation}


\begin{equation}\label{eq:ddefinition}
    \mathbf{D} = \frac{E}{(1+\upsilon)(1-2\upsilon)} \begin{bmatrix}
    1-\upsilon & \upsilon    &  \upsilon & 0 & 0 & 0     \\
      \upsilon &  1-\upsilon &  \upsilon & 0 & 0 & 0     \\
      \upsilon & \upsilon   &  1-\upsilon &  0 & 0 & 0   \\
    0& 0& 0 & \frac{1-2\upsilon}{2} & 0 & 0              \\
    0& 0& 0 & 0 &\frac{1-2\upsilon}{2} & 0               \\
    0& 0& 0 & 0 & 0 &  \frac{1-2\upsilon}{2}             \\
\end{bmatrix}
\end{equation}

onde, $E$ é o módulo de Young e $v$ o módulo de Poisson da rocha.


A EDP da equação \eqref{eq:equilibrio_final} pode então ser escrita em função dos deslocamentos, inserindo \eqref{eq:defor_desloc} e \eqref{eq:hooke} em  \eqref{eq:equilibrio_final}.

\begin{equation}
\label{eq:edp_geomec}
\sopnabla^T \mathbf{D} \sopnabla \deslocamento - \sopnabla^T\alpha \mathbf{m} P_p - \mathbf{f} = 0
\end{equation}

 Essa forma será a utilizada junto dos métodos do elementos finitos para construção de um simulador para regime permanente de geomecânica em duas dimensões. Ao se passar \eqref{eq:edp_geomec} de três dimensões para duas existem duas abordagens possíveis: tensão plana ou deformação plana. As matrizes de elasticidade são apresentadas respectivamente em \eqref{eq:elasticplanestress} e \eqref{eq:elasticplanestrain}.

\begin{equation} \label{eq:elasticplanestress}
\mathbf{D}_{stress} = \frac{E}{1-\upsilon^2}
\begin{bmatrix}
1  & \upsilon & 0 \\
\upsilon & 1 &  0 \\
0 & 0 & \frac{1-\upsilon}{2}
\end{bmatrix}
\end{equation}

\begin{equation} \label{eq:elasticplanestrain}
\mathbf{D}_{strain} = \frac{E}{(1+\upsilon)(1-2\upsilon)}
\begin{bmatrix}
 1-\upsilon & \upsilon    &  0 \\
 \upsilon   &  1-\upsilon &  0 \\
 0& 0 & \frac{1-2\upsilon}{2}
\end{bmatrix}
\end{equation}

De acordo com \citet{jacob}, a condição de deformação plana é melhor aplicada quando o elemento é grosso em relação ao plano xy , que é o caso dos reservatórios de petróleo com os mesmos eixos definidos na Figura \ref{fig:equilibrio}. Dessa forma, artigos como \cite{casteletto}, \cite{planeStrainProblems} e \cite{irina} utilizam a hipótese de deformação plana que é a mesma utilizada nessa dissertação. Os operadores e vetores podem ser redefinidos para os mostrados em \eqref{eq:vetores2d}.

\begin{equation}
\label{eq:vetores2d}
\begin{matrix}
\stef = \begin{bmatrix}
\sxx
\\
\syy
\\
\sxy
\end{bmatrix}
&

;

&

\mathbf{f} = \begin{bmatrix}
f_{x}
\\
f_{y}
\end{bmatrix}
&
;
&

\mathbf{m} = \begin{bmatrix} 1 \\ 1 \\ 0\end{bmatrix}

&
;

&
\sopnabla = \soptwod

&
;

&

\deslocamento = \begin{bmatrix}
u_x
\\
u_y
\end{bmatrix}

\end{matrix}
\end{equation}

Além de \eqref{eq:edp_geomec}, são necessárias as condições de contorno para que o problema fique totalmente definido. Existem dois principais tipos de condição de contorno a de Dirichlet, onde o deslocamento da fronteira é prescrito e condição de contorno de Neumann, onde a tensão normal a fronteira é prescrita. Considerando $\Omega$ o domínio em que está sendo resolvido o problema e $\Gamma = \partial \Omega$ a fronteira do domínio, pode-se dividi-la em duas partes $\Gamma_u$ e $\Gamma_{\sigma}$, onde $\Gamma_u \cup \Gamma_{\sigma} = \Gamma$, $\Gamma_u \cap \Gamma_{\sigma} = \emptyset$, que possuem condição de Dirichlet e Neumann respectivamente. O problema fica então totalmente definido em \eqref{eq:edp_geomec_contorno}.


\begin{equation} \label{eq:edp_geomec_contorno}
\begin{aligned}
     \sopnabla^T \mathbf{D} \sopnabla \deslocamento - \sopnabla^T\alpha \mathbf{m} P_p - \mathbf{f} = 0, & \text{ em }\Omega  \\
     \deslocamento = \bar{\mathbf{u}},& \text{ em } \Gamma_u \\
    \begin{bmatrix}
        \stxx^{\prime\prime} & \stxy^{\prime\prime}
    \end{bmatrix} \mathbf{n} = \bar{t}_x \text{   }     \begin{bmatrix}
        \stxy^{\prime\prime} & \styy^{\prime\prime}
    \end{bmatrix} \mathbf{n} = \bar{t}_y, & \text{   em } \Gamma_{\sigma} 
\end{aligned}  
\end{equation}
onde, $\mathbf{n}$ representa o vetor normal a $\Gamma$, $\bar{\deslocamento} \in \mathbb{R}^2$ e $\bar{\mathbf{t}} = [ \bar{t}_x \quad \bar{t}_y ]$. O problema \eqref{eq:edp_geomec_contorno} pode ser descrito mais genericamente com regiões da fronteira onde uma condição de contorno de Dirichlet é aplicada apenas ao campo $u_x$ ou $u_y$ enquanto o outro campo tem condição de Neumann. Esse exemplo mais geral pode ser encontrado em \citet{hughes}.
