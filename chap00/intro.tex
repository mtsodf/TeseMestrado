

Geomecânica é o estudo das deformações e tensões em solos e rochas, esse estudo se torna de extrema importância para a explotação de campos de petróleo de maneira segura e mais eficiente. De acordo com \citet{ResGeomec}, o estudos de tesões são importantes, pois:

\begin{itemize}
    \item Falhas em poços ocorrem por conta de tensões concentradas ao redor da circunferência dos poços.
    \item Falha geológicas vão deslizar dependendo das tensões e da sua resistência a fricção. 
    \item Depleções do reservatório causam mudanças nas tensões \textit{in-situ} que podem ser maléficas ou benéficas para o reservatório.
\end{itemize}

Além disso, estudos as deformações na rocha causam subsidência do leito marinho que pode tanto causar problemas ambientais como danificar equipamentos de sub-superfície. Injeção demasiadas de água  podem reativar falhas ou provocar fraturas na rocha capeadora fazendo com que aconteçam exsudações na superfície. Dessa forma, fica claro que os estudos de Geomecânica são importantes para uma melhor produção do campo, redução de custos de explotação e também para uma operação mais segura dos campos. E para que estudos geomecânicos complexos sejam realizados são necessárias as simulações numéricas capazes de representar a realidade.

Os modelos geomecânicos, diferentemente dos modelos de fluxo de reservatório, estão interessados em regiões que são não reservatório, por exemplo, o engenheiro pode estar interessado nas tensões em regiões que a trajetória de um novo poço será perfurado, pode estar interessado em analisar se uma injeção demasiada não irá fraturar a rocha capeadora e causar exsudações no fundo do mar ou ainda estar interessado na subsidência do mesmo. Todos esses fenômenos estão associados a regiões não reservatório e, portanto, para essas avaliações é necessário que o domínio de simulação seja maior que as simulações de fluxo pois regiões de \textit{overburden}, \textit{sideburden} e \textit{underburden} devem estar presentes no modelo. A Figura \ref{fig:modelogeomec3d} mostra um corte de um grid de um modelo geomecânico em três dimensões, os elementos pintados de azul são relacionados ao reservatório e pode-se ver que este corresponde a uma pequena parte do modelo. Isso faz com que o número de elementos desses modelos possam ser  grandes tornando necessárias técnicas numéricas e de computação de alto desempenho para que que sejam as simulações possam ser realizadas em tempo viável. Em grande parte, os simuladores passam maior parte do tempo resolvendo sistemas lineares e, portanto, melhorias realizadas nessa parte são bastantes impactantes nos resultados das simulações. Essa dissertação tem como intuito avaliar um pré-condicionador multiescala para reduzir o número de iterações de solver lineares de Krylov, em particular, para o Gradiente Conjugado e Bicgstab. 


% Além disso, os efeitos geomecânicos também são importantes também tem influência no fluxo por conta de modificações no volume poroso e também na permeabilidade do modelo. Conforme mostrado em \citet{rolegeomechanics} as diferenças de produção podem ser consideráveis em uma simulação de fluxo em relação a uma simulação acoplada.

\begin{figure}[!htbp]
\centering
\includegraphics[width=10cm]{chap00/figs/Geresim(0054).png}
\caption{Corte vertical de modelo geomecânico 3D. Células em azul representam o reservatório.}
\label{fig:modelogeomec3d}
\end{figure}

A dissertação é organizada da seguinte forma: no Capítulo \ref{ch:modelagem} são apresentadas as equações que regem os fenômenos geomecânicos, no Capítulo \ref{ch:discretizacao} é apresentado a discretização realizada nas equações do capítulo anterior através do método dos elementos finitos chegando a um sistema linear, no Capítulo \ref{ch:sistemas} são apresentadas estrutura de dados para matrizes esparsas e métodos para solução de sistemas lineares, no Capítulo \ref{ch:multiescala} é apresentado o método multiescala e como ele pode ser utilizado como pré-condicionador para acelerar métodos iterativos de solução dos sistemas lineares, no Capítulo \ref{ch:implementacao} são apresentados detalhes da implementação realizada nesse trabalho e, finalmente, no Capítulo \ref{ch:resultados} são apresentados os experimentos numéricos, onde se destacam a comparação entre pré-condicionador aditivo e multiplicativo e comparação com o método multigrid. 