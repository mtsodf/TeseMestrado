Nesse capítulo serão apresentados detalhes da implementação realizada nesse trabalho. O código foi desenvolvido em Python com auxílio da biblioteca PETSc (\citet{petsc-user-ref}). O PETSc é uma biblioteca famosa com diversas rotinas para computação científica, possui paralelismo utilizando MPI, estruturas de dados para matrizes esparsas, métodos de Krylov para solução de sistemas lineares, conjunto de precondicionadores (em particular os ILU), dentre outros. Estudos da performance da biblioteca podem ser encontrados em \citet{petsc-efficient}. O PETSc foi utilizado com a linguagem Python através dos \textit{bindings} disponibilizados no petsc4py (\citet{Dalcin2011}). O código aqui apresentado foi desenvolvido em serial. 


\section{Estrutura de Classes}

Primeiramente, foi necessário desenvolver uma estrutura para resolver problemas através do método dos elementos finitos, para depois realizar a implementação com o método multiescala. Esse desenvolvimento possui três principais classes: \textbf{Element}, \textbf{Node}, \textbf{FemContext}. As responsabilidades das classes são descritas abaixo:

\begin{itemize}
    \item \textbf{Node}: essa classe tem como intuito representar os nós do problema de elementos finitos. Guarda como propriedade as coordenadas x, y correspondentes, uma numeração global associada ao grid fino e se o nó pertence ou não a uma fronteira com condição de Dirichlet.
    \item \textbf{Element}: essa classe tem como intuito representar os elementos do problema de elementos finitos. Tem referência para os nós que pertencem ao elemento, guarda os valores do coeficiente de Young e Poisson, sabe calcular as funções de forma em coordenadas locais conforme \eqref{eq:definicaofuncform} e também é responsável por calcular a matriz do elemento local conforme \eqref{eq:matrizelemento}.
    \item \textbf{FemContext}: essa classe tem como intuito representar um problema de elementos finitos. Ela possui o conjunto de elementos e nós do problema em questão. Dá a numeração de cada um dos graus de liberdade dos nós. Calcula a matriz de rigidez através do Algoritmo \ref{alg:buildmatrix} e o lado direito correspondente. 
\end{itemize}

Essas três classes conjuntamente são capazes de montar e solucionar um problema de elementos finitos. Um fato importante é que a classe \textbf{FemContext} é responsável pela numeração dos graus de liberdade, pois, para calcular as funções de base, diferentes problemas locais tem que ser resolvidos e diferentes numeração locais são necessárias. Em uma implementação mais clássica de elementos finitos, o mais natural seria a numeração do grau de liberdade estar associada com a classe \textbf{Node}.


\section{Montagem dos Operador Multiescala}

Para a implementação do método multiescala, foi criada a classe \textbf{Multiscale}, essa classe tem como responsabilidade receber um objeto \textbf{FemContext} que representa o grid fino e calcular os operadores de prolongamento ($\mathbf{P}$), restrição ($\mathbf{P}^T$) e operador grosso ($\rigidmatrixcoarse$).


O primeiro passo para utilizar o método multiescala é a escolha do nível de engrossamento. A implementação aceita diferentes fatores para a direção x e y, apesar dos testes serem apresentados com o mesmo fator. O método \textbf{CoarseContext} da classe \textbf{Multiscale} é responsável por realizar a tarefa de construir $\mathbf{P}$ e $\rigidmatrixcoarse$. Essa tarefa é dividida nos seguintes passos: 

\begin{itemize}
    \item Criar objetos \textbf{FemContext} para cada um dos elementos grossos $\coarseelement$.
    \item Adicionar os nós e elementos correspondentes em cada um dos seus respectivos \textbf{FemContext}
    \item Para cada \textbf{FemContext} calcular a matriz de rigidez $\rigidmatrixelementms$.
    \item Resolver os oito problemas associados as funções de base encontrando $\basefunctionelemcoarse[1],  \basefunctionelemcoarse[2], \hdots, \basefunctionelemcoarse[8]$.
    \item Fazer o Assembly do operador de prolongamento.
    \item Calcular $\rigidmatrixcoarse = \mathbf{P}^T \rigidmatrix \mathbf{P}$
\end{itemize}

A Figura \ref{fig:esquemaconstrucao} apresenta um esboço da construção desse operador. Pode-se perceber que os problemas de cada \textbf{FemContext} são independentes, exceto pela solução dos problemas das arestas em comum. A implementação desse trabalho resolveu as arestas da fronteira duas vezes, uma implementação mais otimizada pode resolver apenas uma vez cada problema de fronteira. Em relação a resolver os oito problemas locais, foi realizada a fatoração LU da matriz que foi reaproveitada para a solução desses problemas. 

\begin{figure}[!htbp]
\centering
\includegraphics[width=\textwidth]{chap07/figs/esquemaprolongamento.png}
\caption{Esquema de construção dos operadores grossos. Mostrando um grid 8x8 sendo descomposto em um grid grosseiro 2x2 onde cada elemento grosso é formado por 4x4 elementos finos.}
\label{fig:esquemaconstrucao}
\end{figure}


O cálculo das matrizes dos operadores locais $\rigidmatrixelementms$ tem os mesmos coeficientes presentes na matriz de rigidez $\rigidmatrix$ visto que o operador de \eqref{eq:edp_geomec} é o mesmo de \eqref{eq:operadorlocal}. Assim, para o código executar mais rápido a matriz de rigidez de cada elemento foi armazenada para não precisar ser calculada novamente, isso torna o custo da memória maior, pois esses valores não precisam ser necessariamente guardados. Outra opção é perceber que a matriz de um operador local é uma submatriz do operador original e, portanto, poderia ser utilizado o método \textbf{GetSubMatrix} das matrizes do Petsc para construir $\rigidmatrixelementms$ a partir de $\rigidmatrix$ conforme mostra a Figura \ref{fig:submatrix}. Essa ideia é similar a utilizar uma reordenação Wire-Basket como apresentada em \cite{casteletto} e \cite{irina}.

\begin{figure}[!htbp]
\centering
\includegraphics[width=\textwidth]{chap07/figs/submatrix.png}
\caption{Exemplo de construção de uma matriz relacionada a um problema local através do \textbf{GetSubMatrix}.}
\label{fig:submatrix}
\end{figure}


A classe \textbf{Multiscale} também possui um método solve, que só pode ser chamado após o método \textbf{CoarseContext} ter sido executado. Esse método recebe como entrada um vetor $\mathbf{f}^h$ relacionado a um lado direito do grid fino e resolve o problema no grid grosso e retorna para o grid fino. Em outras palavras, esse método retorna o seguinte valor $\mathbf{P}(\rigidmatrixcoarse)^{-1}\mathbf{P}^T \mathbf{f}^h$. Para a solução do sistema $(\rigidmatrixcoarse)^{-1}$ a solução utilizada foi uma fatoração LU. Essa fatoração é guardada para ser utilizada entre as iterações do método de Krylov quando o objeto o método \textbf{Multiescala} está sendo utilizado como precondicionador.

                                                                                   